\documentclass{article}

\usepackage[brazil]{babel}
\usepackage{lipsum}
\usepackage{indentfirst}

\title{Projeto Final Lia (Nome Provisório)}

\author{Jonathas dos Santos \and Lucas Vinícius de Lima Assis}

\begin{document}
    \maketitle
    
    \section{Descrição}
        % Lucas
        Nosso projeto é uma simulação que se utiliza da tecnologia de
        \emph{Processamento de Linguagem Natural} (NPL) para interação entre
        usuário e programa, com a finalidade de auxiliar no aprendizado de uma
        língua estrangeira. Nosso programa simula, por meio do
        Pygame\footnote{Pygame é uma biblioteca python, voltada para a criação
        de jogos, construída sobre SDL trazendo funcionalidades e suporte a
        gráficos, manipulação de imagens, áudio, leitura de teclado e mouse,
        entre outros.}, uma barraquinha de feira com dois vendedores
        estrangeiros, que contém dos mais diversos produtos a venda, desde
        frutas a eletrônicos. O usuário sabe que os vendedores não conhecem sua
        língua, e como o educado cliente que é, interage com eles em sua língua
        natal. 

        Sobre a simulação, para iniciá-la, o usuário deve, por meio da fala, dar
        boas vindas aos vendedores no idioma que eles conhecem. Em seguida,
        pode lhes perguntar quais produtos estão disponíveis, ou se algum
        produto em específico está disponível. Pode fazer perguntas referente ao
        preço de certo produto, quantos produtos estão em estoque, ou até mesmo
        se há alguma encomenda especial. Quando o usuário decidir pagar por seus
        produtos, o vendedor lhe dirá, em seu idioma, o valor da compra, e o
        usuário deverá entregar-lhe a quantidade correta. Após comprado o que
        desejava, o usuário, educado que é, dá uma despedida aos vendedores, e
        segue seu passeio na feira.

        A escolha desse cenário foi feita com o intuito de criar uma situação em que o
        usuário tenha contato com diversas palavras e expressões da língua. Na
        cena descrita, houve o contato com expressões para se iniciar uma
        conversa, para se despedir, o uso do modo interrogativo para saber quais
        produtos estão em estoque, o uso do modo afirmativo ou imperativo para
        se escolher quais produtos deseja, o contato com os diversos nomes de
        produtos, e o contato com números na hora do pagamento. 

        

    \section{Passos Envolvidos}

    \subsection{Preparação dos Dados}

    \subsection{Construção do Modelo}

    \subsection{Treinamento do Modelo}

    \subsection{Ajuste fino do Modelo}

    \subsection{Implantação do Modelo}

    \section{Código do Projeto}

\end{document}
